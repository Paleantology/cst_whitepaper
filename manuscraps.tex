%These limitations have caused instructors to seek other technologies to help students learn.
%\rs{I think this whole paragraph is either redundant or contains detail that is best discussed in the detailed sections below}
%\aw{I think you're right. It made more sense before we reorganized, and before the detials were filled in}
%Here we review various solutions to supporting students' computing needs,
%discuss when each platform might be more useful and appropriate,
%and outline methods use to teach computing effectively and inclusively to biology students.
%These solutions can be broadly grouped into local computing platforms and remote computing platforms.
%Local computing platforms involve learners installing or working with personal or local machines as directed by an instructor.
%Typically, in these setups, the learner is responsible for managing their own machine.
%In this category, this review will discuss having students install software on their personal computers and using Raspberry Pi to set up small-scale bioinformatics machines. 
%These computing platforms offer learners the comfort of working on their own machine, and allow them to exit the course with a set of software for use in their own research. 
%Remote computing platforms involve students interacting with a server or other machine, typically off-site, that is not under control of the learner.
%In this case, the instructor or support staff typically sets up and manages the server before and during the course.
%These computing platforms minimize the amount of management the learner must do in terms of installing software and preparing their computer for class,
%although students may experience an unfamiliar interface and will also have limited control over the platform.
%Examples of this type of solution that we will discuss are RStudio Server and JupyterHub. 

%Collectively called computing platforms, 
%these refer to platforms that enable the instructor to demonstrate programming concepts, distribute course materials (notes, homeworks), and assess student outcomes.
%Examples of course service technologies include RStudio Server, JupyterHub, and simpler strategies such as having learners install software on their personal computers.
%Are computing platform just servers or all IDEs as well?
%\rs{I commented out some details here as we'll discuss them later}


%Thus, there is significant need for an understanding of resources to support computational instruction. %does this need citation? Probably - if anyone has one, feel free to add.
%
%to collectively refer to options ranging from local computers to shared cloud servers..
%Here we discuss these options, provide suggestions for when each approach can server the needs of the instructor and students, as well as 
%In addition, as in most fields of study, computational biology \rs{need an alternate phrase - this is narrow}
%outline a variety of methods for teaching computational content.
%, such as interactive tutorial sessions (live coding) and student-centered cooperative learning. 